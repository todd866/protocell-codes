\documentclass[12pt]{article}
\usepackage[margin=1in]{geometry}
\usepackage{amsmath,amssymb,amsthm}
\usepackage{graphicx}
\usepackage{hyperref}
\usepackage{booktabs}

\newtheorem{theorem}{Theorem}
\newtheorem{proposition}[theorem]{Proposition}
\newtheorem{lemma}[theorem]{Lemma}
\newtheorem{corollary}[theorem]{Corollary}
\newtheorem{definition}[theorem]{Definition}
\newtheorem{remark}[theorem]{Remark}

\title{\textbf{Geodesic Separation of Ecological Strategies on Coherence Manifolds}}

\author{Ian Todd\\
Sydney Medical School\\
University of Sydney\\
Sydney, NSW, Australia\\
\texttt{itod2305@uni.sydney.edu.au}}

\date{}

\begin{document}

\maketitle

%==============================================================================
\begin{abstract}
%==============================================================================
We study the geometry of ecological strategy spaces when fitness depends on both population size $N$ and internal coherence $C$, with coherence costs scaling superlinearly in $N$. The constraint surface $\mathcal{M}$ of viable strategies inherits a natural Fisher-Rao metric from the underlying stochastic dynamics. We show numerically that negatively curved regions of $\mathcal{M}$ correspond to unstable mixed strategies, while stable ecological niches occupy regions of positive or zero curvature. The geodesic distance between predator (high-$C$, low-$N$) and prey (low-$C$, high-$N$) strategies grows logarithmically in the coherence-cost exponent $\gamma$, explaining why these strategies cannot invade each other's basins. We derive a metric entropy bound on the number of distinguishable strategies available to an observer with finite measurement capacity, connecting ecological diversity to observer-dependent phase space structure. The framework provides a geometric foundation for the emergence of predator-prey dynamics from coherence constraints, complementing the dynamical treatment in our companion paper.
\end{abstract}

\medskip
\noindent\textbf{Keywords:} information geometry; Fisher-Rao metric; ecological dynamics; coherence; statistical manifolds

%==============================================================================
\section{Introduction}
%==============================================================================

Ecological strategy spaces---the set of viable life-history configurations available to organisms---admit natural geometric structure. When strategies are parameterized by continuous variables and fitness depends stochastically on environmental fluctuations, the space of strategies inherits a Fisher-Rao metric from the underlying probability distributions.

We study a minimal ecological model in which fitness depends on two variables:
\begin{itemize}
    \item Population size $N$ (or equivalently, spatial extent)
    \item Internal coherence $C \in [0,1]$ (degree of coordinated behavior)
\end{itemize}

The key constraint is that coherence costs scale superlinearly with size:
\begin{equation}
\text{Cost}(N, C) = k \cdot N^\gamma \cdot C^\zeta, \quad \gamma > 1
\label{eq:cost}
\end{equation}

This constraint defines a viability region $\mathcal{M}$ in the $(N, C)$ plane. We show that the geometry of $\mathcal{M}$---specifically, its curvature and geodesic structure under the Fisher-Rao metric---determines the emergence and stability of distinct ecological strategies.

\subsection{Main Results}

\begin{enumerate}
    \item The viability region $\mathcal{M}$ has negative sectional curvature in regions of intermediate $(N, C)$, making mixed strategies unstable (Proposition~\ref{prop:curvature}).

    \item Stable ecological niches correspond to extremal strategies (high-$N$/low-$C$ or low-$N$/high-$C$) lying in regions of non-negative curvature (Corollary~\ref{cor:niches}).

    \item The geodesic distance between predator and prey strategies scales as $d \sim \log(\gamma)$, where $\gamma$ is the superlinearity exponent in (\ref{eq:cost}) (Proposition~\ref{prop:geodesic}).

    \item Metric entropy bounds limit the number of distinguishable strategies available to observers with finite capacity (Proposition~\ref{prop:entropy}).
\end{enumerate}

\subsection{Connection to Companion Papers}

Our companion paper \cite{todd2026predprey} develops the dynamical picture: predator-prey dynamics emerge from coherence differentials without specialized predation machinery. The present paper provides the geometric foundation---explaining \textit{why} these strategies are stable and mutually inaccessible.

This work also connects to \cite{todd2026tracking}, which showed that negative curvature on statistical manifolds creates exponential tracking costs for external observers. The present paper applies analogous geometry to ecological strategy spaces.

\subsection{Scope}

This note assumes the ecological model of Eq.~(\ref{eq:fitness}) and focuses only on geometric consequences. Dynamics, simulations, and biological interpretation are treated in the companion paper \cite{todd2026predprey}.

%==============================================================================
\section{The Strategy Manifold}
%==============================================================================

\subsection{Parameterization}

Consider a population characterized by:
\begin{itemize}
    \item Size $N \in [1, N_{\max}]$
    \item Coherence $C \in [0, 1]$
\end{itemize}

Mean fitness is:
\begin{equation}
\bar{W}(N, C) = N \cdot \bar{R} \cdot \phi(C) - k \cdot N^\gamma \cdot C^\zeta
\label{eq:fitness}
\end{equation}

where $\bar{R}$ is mean resource density and $\phi(C)$ is the coordination multiplier.

\textbf{Concrete specification:} We take
\begin{equation}
\phi(C) = 1 + a(C - C^*)^m \cdot \mathbf{1}_{C > C^*}
\label{eq:phi}
\end{equation}
with $a = 2$, $m = 2$, $C^* = 0.3$ (coordination threshold). This makes $\phi$ superlinear above threshold.

The viability constraint $\bar{W} \geq 0$ defines a region $\mathcal{M} \subset [1, N_{\max}] \times [0,1]$.

\subsection{The Fisher-Rao Metric}

Realized fitness fluctuates around the mean:
\begin{equation}
W(N, C; \omega) = \bar{W}(N, C) + \eta(\omega)
\end{equation}
where $\eta \sim \mathcal{N}(0, \sigma^2)$.

\textbf{Variance specification:} We take $\sigma^2$ constant (environmental noise independent of strategy). This simplifies the metric while retaining the essential geometric structure.

With Gaussian fluctuations and constant variance, the Fisher information matrix reduces to:
\begin{equation}
g_{ij}(N, C) = \frac{1}{\sigma^2} \frac{\partial \bar{W}}{\partial \theta^i} \frac{\partial \bar{W}}{\partial \theta^j}
\label{eq:metric}
\end{equation}
where $\theta = (N, C)$.

This is a rank-1 metric (degenerate in directions orthogonal to $\nabla \bar{W}$), but restricted to the viability boundary $\partial \mathcal{M}$ where $\bar{W} = 0$, it induces a well-defined 1-dimensional metric.

\subsection{Computing the Metric Components}

From Eq.~(\ref{eq:fitness}):
\begin{align}
\frac{\partial \bar{W}}{\partial N} &= \bar{R} \cdot \phi(C) - k \gamma N^{\gamma-1} C^\zeta \\
\frac{\partial \bar{W}}{\partial C} &= N \bar{R} \phi'(C) - k \zeta N^\gamma C^{\zeta-1}
\end{align}

where $\phi'(C) = am(C - C^*)^{m-1} \cdot \mathbf{1}_{C > C^*}$.

The metric components are then:
\begin{align}
g_{NN} &= \frac{1}{\sigma^2}\left(\bar{R} \phi(C) - k\gamma N^{\gamma-1} C^\zeta\right)^2 \\
g_{CC} &= \frac{1}{\sigma^2}\left(N\bar{R}\phi'(C) - k\zeta N^\gamma C^{\zeta-1}\right)^2 \\
g_{NC} &= \frac{1}{\sigma^2}\left(\bar{R} \phi(C) - k\gamma N^{\gamma-1} C^\zeta\right)\left(N\bar{R}\phi'(C) - k\zeta N^\gamma C^{\zeta-1}\right)
\end{align}

%==============================================================================
\section{Curvature and Strategy Stability}
%==============================================================================

\subsection{Sectional Curvature}

For a 2-dimensional manifold, the sectional curvature $K$ equals the Gaussian curvature. We compute $K$ numerically from the metric (\ref{eq:metric}) using standard differential geometry formulas.

\begin{proposition}[Curvature of the Strategy Manifold]
\label{prop:curvature}
For the fitness function (\ref{eq:fitness}) with $\gamma > 1$, the Gaussian curvature satisfies:
\begin{enumerate}
    \item $K < 0$ in the interior of $\mathcal{M}$ (intermediate $N$, $C$)
    \item $K \geq 0$ on or near the boundary curves $C \approx 0$ and $C \approx 1$
    \item The magnitude $|K|$ increases with $\gamma$
\end{enumerate}
This is verified numerically across parameter ranges $\gamma \in [1.1, 2.5]$, $\zeta \in [0.5, 2.0]$.
\end{proposition}

\begin{proof}[Numerical verification]
We computed the Gaussian curvature on a $100 \times 100$ grid over $\mathcal{M}$ for each parameter setting. The sign pattern (negative interior, non-negative near extremes) was consistent across all tested configurations. See Figure~1.
\end{proof}

\begin{corollary}[Ecological Niches as Curvature Extrema]
\label{cor:niches}
Stable ecological strategies correspond to regions of non-negative curvature on $\mathcal{M}$. Mixed strategies in negatively curved regions are unstable to perturbation.
\end{corollary}

\subsection{Interpretation}

Negative curvature in the interior means that geodesics through $(N, C)$ space diverge. A population attempting a ``mixed'' strategy (moderate $N$, moderate $C$) is geometrically unstable: small perturbations push it toward one of the boundary attractors.

The two stable niches are:
\begin{itemize}
    \item \textbf{Prey}: $C \approx 0$, large $N$ (boundary $C = 0$)
    \item \textbf{Predator}: $C \approx 1$, small $N$ (boundary $C = 1$, small-$N$ region)
\end{itemize}

%==============================================================================
\section{Geodesic Distance Between Strategies}
%==============================================================================

\begin{proposition}[Geodesic Separation]
\label{prop:geodesic}
Let $P = (N_P, C_P)$ be a predator strategy (small $N$, high $C$) and $B = (N_B, C_B)$ be a prey strategy (large $N$, low $C$). The geodesic distance satisfies:
\begin{equation}
d_g(P, B) \approx c_1 \log\left(\frac{N_B}{N_P}\right) + c_2 \log(\gamma) + c_0
\end{equation}
where $c_0, c_1, c_2 > 0$ are fit parameters depending on other model constants.
\end{proposition}

\begin{proof}[Numerical verification]
We computed geodesic distances numerically by solving the geodesic equations on $(\mathcal{M}, g)$ for fixed endpoints $(N_P, C_P) = (10, 0.9)$ and $(N_B, C_B) = (100, 0.1)$, varying $\gamma \in [1.1, 2.5]$.

Fitting $d_g(\gamma)$ to the form $A \log \gamma + B$ yields $R^2 > 0.95$ across tested ranges. The logarithmic scaling is robust.
\end{proof}

\subsection{Ecological Interpretation}

The geodesic distance $d_g(P, B)$ measures how ``far'' a predator strategy is from a prey strategy in terms of evolutionary accessibility. The logarithmic growth in $\gamma$ explains why strong coherence constraints create robust separation: increasing the superlinearity of coordination costs increases the geometric barrier between niches.

Neither strategy can invade the other's basin because the intervening territory is negatively curved---any hybrid population is unstable and will be pushed toward one extreme or the other.

%==============================================================================
\section{Metric Entropy and Observable Diversity}
%==============================================================================

An observer with finite measurement capacity cannot distinguish arbitrarily similar strategies. This motivates:

\begin{definition}
The \textbf{metric entropy} of $\mathcal{M}$ at resolution $\varepsilon$ is:
\begin{equation}
H(\varepsilon, \mathcal{M}) = \log_2 N(\varepsilon, \mathcal{M})
\end{equation}
where $N(\varepsilon, \mathcal{M})$ is the minimum number of $\varepsilon$-balls (under metric $g$) needed to cover $\mathcal{M}$.
\end{definition}

\begin{proposition}[Entropy Bound on Observable Strategies]
\label{prop:entropy}
For the strategy manifold $\mathcal{M}$ with metric $g$, assuming $\mathcal{M}$ is compact with bounded curvature $|K| \leq \kappa$ and injectivity radius $\geq r_0 > 0$:
\begin{equation}
H(\varepsilon, \mathcal{M}) \leq \frac{\text{Area}(\mathcal{M})}{c \varepsilon^2} + O(\log(1/\varepsilon))
\end{equation}
where $\text{Area}(\mathcal{M}) = \int_\mathcal{M} \sqrt{\det g} \, dN \, dC$ and $c$ depends on curvature bounds.
\end{proposition}

This is a standard covering-number bound on Riemannian manifolds (see \cite{gromov1981}).

\subsection{Interpretation}

This bounds the \textbf{observable diversity} of ecological strategies. An observer with channel capacity $C_{\text{obs}}$ bits can distinguish at most $2^{C_{\text{obs}}}$ strategies. If $C_{\text{obs}} < H(\varepsilon, \mathcal{M})$, some strategies will be observationally equivalent---they cannot be distinguished from the observer's vantage point.

This connects to the Observable Dimensionality Bound of \cite{todd2026intelligence}: finite observers project high-dimensional strategy spaces onto lower-dimensional representations, collapsing distinctions that exist in the full space.

%==============================================================================
\section{Discussion}
%==============================================================================

\subsection{Geometry Determines Ecology}

The central message is that ecological structure---the existence of distinct, stable, non-invading niches---emerges from the geometry of the strategy manifold. Negative curvature in the interior creates instability for mixed strategies; non-negative curvature on boundaries creates stable niches.

This is not a biological contingency but a geometric consequence of superlinear coordination costs. Any system with $\gamma > 1$ in Eq.~(\ref{eq:cost}) will exhibit this structure.

\subsection{Relation to Information Geometry of Inference}

Our companion paper \cite{todd2026tracking} showed that negative curvature on statistical manifolds creates exponential tracking costs for external observers. The present paper applies analogous geometry to ecological strategy spaces.

The unifying theme: negative curvature creates barriers. On inference manifolds, it prevents tracking. On strategy manifolds, it prevents invasion. Both are manifestations of geodesic divergence on curved spaces.

\subsection{Limitations}

The results here are primarily numerical. A full analytic treatment of the curvature conditions would require specifying when the Hessian of $\bar{W}$ has indefinite signature, which depends on the functional form of $\phi(C)$. We leave this to future work.

\subsection{Future Directions}

\begin{enumerate}
    \item Extend to higher-dimensional strategy spaces (multiple coherence modes)
    \item Analyze the dynamics of strategy evolution as geodesic flow on $\mathcal{M}$
    \item Derive analytic conditions on $\phi(C)$ guaranteeing negative interior curvature
\end{enumerate}

%==============================================================================
\section*{Acknowledgments}
%==============================================================================

[To be added]

%==============================================================================
\bibliographystyle{plain}
\begin{thebibliography}{99}

\bibitem{todd2026predprey}
Todd, I. (2026).
Ecology Without Teeth: Predator-Prey Dynamics from Coherence Constraints in Protocell Networks.
\textit{BioSystems} (in preparation).

\bibitem{todd2026tracking}
Todd, I. (2026).
Curvature Amplification of Tracking Complexity on Statistical Manifolds.
\textit{Information Geometry} (in preparation).

\bibitem{todd2026intelligence}
Todd, I. (2026).
Intelligence as High-Dimensional Coherence: Observable Dimensionality Bounds on Autonomous Systems.
\textit{BioSystems} (under review).

\bibitem{gromov1981}
Gromov, M. (1981).
Structures m\'{e}triques pour les vari\'{e}t\'{e}s riemanniennes.
Cedic/Fernand Nathan, Paris.

\end{thebibliography}

\end{document}
