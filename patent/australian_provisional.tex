\documentclass[12pt]{article}
\usepackage{amsmath, amssymb}
\usepackage{graphicx}
\usepackage{geometry}
\usepackage{booktabs}
\geometry{margin=1in}

\begin{document}

\begin{center}
{\LARGE\bfseries AUSTRALIAN PROVISIONAL PATENT APPLICATION}\\[1.5em]

{\Large\bfseries Method and System for Generating Digital Codes from Coupled Compartment Networks via Multi-Channel Coordination}\\[2em]

\textbf{Applicant:} Ian Todd\\
Coherence Dynamics Australia Pty Ltd\\
Sydney, NSW, Australia\\
ian@coherencedynamics.com\\[1em]

\textbf{Inventor:} Ian Todd\\[1em]

\textbf{Date:} January 2026
\end{center}

\vspace{1em}

\begin{abstract}
A method and system for generating digital codes from abiotic chemistry is disclosed. The invention exploits coordination equilibria between coupled chemical compartments communicating via multiple signaling channels. The system comprises: (a) an array of chemical compartments with high-dimensional internal dynamics; (b) multi-channel coupling between compartments including bioelectric, chemical, mechanical, and/or redox signaling; (c) external stimulus triggering coordination dynamics; and (d) readout of stable collective patterns as digital symbols. Codes emerge as game-theoretic equilibria---stable patterns that allow compartments to coordinate without knowing each other's full internal state. Applications include abiotic information processing, origin-of-life research, unconventional computing, and physical unclonable functions.
\end{abstract}

\tableofcontents
\newpage

%==============================================================================
\section{Technical Field}
%==============================================================================

The invention relates to methods and systems for generating digital information from chemical processes, specifically methods for producing discrete, reproducible codes from coordination dynamics between coupled chemical compartments without biological components or pre-programmed symbol assignments.

%==============================================================================
\section{Background}
%==============================================================================

\subsection{The Code Generation Problem}

A fundamental question in origin-of-life research is how digital, symbolic information can arise from continuous chemical processes. Traditional approaches seek specific ``coding molecules'' or attempt to engineer discrete states directly.

These approaches face several challenges:
\begin{itemize}
\item Chemistry is fundamentally continuous (concentrations, reaction rates)
\item Discrete states must be artificially imposed
\item Pre-programmed symbol assignments raise questions of circularity
\item Isolated systems lack the stability required for reproducible codes
\end{itemize}

\subsection{The Coordination Insight}

We have discovered that discrete codes emerge naturally when:
\begin{enumerate}
\item Multiple compartments with high-dimensional internal dynamics are weakly coupled
\item Compartments communicate via multiple signaling channels (bioelectric, chemical, mechanical, redox)
\item External stimulus drives the network away from equilibrium
\item The network settles into stable coordination patterns
\item These patterns are readable as discrete symbols
\end{enumerate}

The key insight is that \textbf{codes are coordination equilibria, not transmitted messages}. A ``symbol'' is not something sent from encoder to decoder---it is a stable pattern that allows coupled compartments to influence each other's behavior without requiring full state knowledge.

\subsection{Biological Precedent}

Levin and colleagues have demonstrated that bioelectric patterns carry morphogenetic ``codes'' in biological development---voltage gradients that specify tissue identity. This shows that multi-channel coordination between cells is a fundamental mechanism for information processing in living systems.

We propose that this capability is not derived but \textit{primitive}: coupled protocell networks coordinating via boundary signals before genetic encoding existed. The genetic code is a late optimization---compression of codes that already existed as coordination interfaces.

\subsection{Prior Art}

Existing approaches to abiotic code generation include:
\begin{itemize}
\item \textbf{Oscillator networks}: Coupled oscillators produce phase states, but typically require precise tuning and produce limited alphabets
\item \textbf{Reservoir computing}: Uses high-dimensional dynamics but requires external training
\item \textbf{Autocatalytic sets}: Produce self-sustaining reactions but do not explain discrete symbol formation
\item \textbf{Single-vessel projection systems}: Project high-D dynamics onto boundaries but lack the robustness of multi-compartment coordination
\end{itemize}

There is a need for a method that generates digital codes from game-theoretic coordination between coupled compartments, exploiting multi-channel signaling for robustness and temporal separation.

%==============================================================================
\section{Summary of the Invention}
%==============================================================================

The invention provides methods and systems for generating digital codes from abiotic chemistry through coordination dynamics between coupled compartments.

In one aspect, the invention provides a method comprising:
\begin{enumerate}
\item[(a)] Providing an array of chemical compartments, each containing high-dimensional reaction dynamics;
\item[(b)] Establishing multi-channel coupling between compartments via one or more of: bioelectric signaling (ion gradients, membrane potentials), chemical signaling (diffusible molecules), mechanical signaling (membrane tension), or redox signaling (electron transfer);
\item[(c)] Applying external stimulus to drive the network away from equilibrium;
\item[(d)] Allowing the network to settle into a stable coordination pattern; and
\item[(e)] Reading the collective pattern as a digital symbol or symbol sequence.
\end{enumerate}

In another aspect, the invention provides a system comprising:
\begin{enumerate}
\item[(a)] An array of chemical compartments with high-dimensional internal dynamics;
\item[(b)] Multi-channel coupling mechanisms between compartments;
\item[(c)] Stimulus delivery apparatus;
\item[(d)] Pattern readout apparatus configured to detect collective coordination states; and
\item[(e)] Symbol assignment based on stable pattern identity.
\end{enumerate}

%==============================================================================
\section{Brief Description of the Drawings}
%==============================================================================

\textbf{FIG.~1}: System architecture showing array of coupled compartments, multi-channel signaling, and collective pattern readout.

\textbf{FIG.~2}: Multi-channel coupling schematic showing bioelectric (fast), chemical (slow), mechanical (fast), and redox (medium) signaling between compartments.

\textbf{FIG.~3}: Coordination equilibria: multiple stable collective patterns corresponding to different symbols.

\textbf{FIG.~4}: Encoding table from experiments showing stimulus configurations mapped to stable output patterns.

\vspace{0.5em}
\noindent\textit{Note: Figures are based on numerical simulations demonstrating the coordination-code mechanism.}

%==============================================================================
\section{Detailed Description of the Invention}
%==============================================================================

\subsection{Theoretical Foundation}

\subsubsection{Coordination Equilibria}

Consider an array of $N$ compartments, each with internal state $\mathbf{x}_i(t) \in \mathbb{R}^n$. Each compartment can ``see'' only the boundary states of its neighbors---low-dimensional projections $\mathbf{b}_i = \pi(\mathbf{x}_i)$.

The dynamics of each compartment depend on both internal chemistry and neighbor boundary signals:
\begin{equation}
\frac{d\mathbf{x}_i}{dt} = \mathbf{f}_i(\mathbf{x}_i) + \sum_{j \in \mathcal{N}(i)} \mathbf{g}_{ij}(\mathbf{b}_j) + \mathbf{h}_i(S(t))
\end{equation}
where $\mathcal{N}(i)$ is the neighborhood of compartment $i$, $\mathbf{g}_{ij}$ is the response to neighbor $j$'s boundary signal, and $\mathbf{h}_i(S(t))$ is the response to external stimulus.

A \textbf{coordination equilibrium} is a collective state $\{\mathbf{x}_i^*\}$ such that:
\begin{enumerate}
\item Each compartment's boundary signal is consistent with its internal state
\item Each compartment's internal state is stable given its neighbors' boundary signals
\item The collective pattern is stable under small perturbations
\end{enumerate}

This is a game-theoretic equilibrium: each compartment is ``playing'' the best response to its neighbors, and no compartment has incentive to deviate.

\subsubsection{Multi-Channel Signaling}

Compartments communicate via multiple channels with different characteristics:

\begin{center}
\begin{tabular}{llll}
\toprule
\textbf{Channel} & \textbf{Mechanism} & \textbf{Timescale} & \textbf{Function} \\
\midrule
Bioelectric & Ion flux, membrane potential & ms--s & Fast coordination \\
Chemical & Diffusible molecules & s--min & Metabolic context \\
Mechanical & Membrane tension & ms--s & Physical coupling \\
Redox & Electron transfer & s--min & Energy state \\
\bottomrule
\end{tabular}
\end{center}

Multi-channel signaling provides:
\begin{itemize}
\item \textbf{Temporal separation}: Fast channels for synchronization, slow channels for context
\item \textbf{Redundancy}: Multiple pathways reduce error rates
\item \textbf{Cross-validation}: Inconsistent signals across channels indicate instability
\end{itemize}

\subsubsection{Emergent Digitality via Substrate Competition}

Discrete symbols emerge from the combination of:
\begin{itemize}
\item High-dimensional internal dynamics (many degrees of freedom per compartment)
\item Low-dimensional boundary readout (each compartment sees only projections)
\item \textbf{Substrate competition (lateral inhibition)}: Output channels within each compartment compete for shared metabolic fuel. Strong channels suppress weak channels, forcing winner-take-all dynamics that collapse continuous outputs into discrete binary states
\item Threshold physics (voltage-gated transitions, precipitation, phase changes)
\item Multiple stable equilibria (different coordination patterns)
\end{itemize}

The substrate competition mechanism is critical: it provides emergent discretization without external logic. The digital nature of the code arises from reaction dynamics, not from designed thresholds or pre-programmed classification.

\subsubsection{Alternative Digitization Mechanisms}

While substrate competition (lateral inhibition via mass-action kinetics) is the preferred embodiment, the invention encompasses alternative physical mechanisms for discretization:

\textbf{Precipitation thresholds}: Compartments contain species that precipitate above a concentration threshold (e.g., calcium phosphate, barium sulfate). The precipitate/dissolved state provides binary readout. Neighbor coupling occurs via shared ion pools.

\textbf{Phase separation}: Compartments contain polymer mixtures near phase separation boundaries (e.g., PEG/dextran aqueous two-phase systems). Small perturbations trigger discrete phase transitions. The phase state (single-phase vs. two-phase) provides binary output.

\textbf{Voltage-gated transitions}: Ion channels or synthetic ionophores with steep voltage dependence (e.g., alamethicin pores) provide threshold switching. Membrane potential above/below threshold gives discrete conductance states.

\textbf{Autocatalytic bistability}: Chemical systems with positive feedback (e.g., iodate-arsenite, chlorite-iodide) exhibit bistable steady states. Coupling via shared reactants creates coordinated switching.

\textbf{pH-driven conformational switches}: Peptides or synthetic molecules with sharp pH-dependent conformational transitions (e.g., pH-responsive polymers) provide discrete structural states.

Each mechanism provides emergent discretization from physical dynamics without programmed logic. The claims are not limited to substrate competition; any physical mechanism producing stable discrete states from continuous dynamics falls within the scope of the invention.

The number of distinguishable symbols is determined by the number of stable coordination equilibria accessible under different stimulus conditions. \textbf{Spatial heterogeneity} in the stimulus field (e.g., center-edge light gradients, top-bottom temperature gradients) breaks degeneracy between configurations, ensuring each input maps to a unique output.

\subsection{First Embodiment: Vesicle Array Code Generator}

\subsubsection{Structure}

The apparatus comprises:
\begin{itemize}
\item Microfluidic chip with hexagonal array of wells (e.g., 19 wells in honeycomb pattern)
\item Lipid vesicles (synthetic DPPC/DOPE from petrochemical feedstocks, 10 $\mu$m diameter) loaded into each well
\item Oscillatory chemistry inside vesicles (BZ-type reaction mixture)
\item Synthetic ionophores (e.g., crown ethers, calixarenes, or chemically synthesized valinomycin from non-biological feedstocks) for bioelectric coupling
\item Shared aqueous medium for chemical coupling
\item Membrane contact between adjacent vesicles for mechanical coupling
\item UV lamp and temperature controller for stimulus delivery
\item Multi-channel readout: pH dye fluorescence, membrane potential (voltage-sensitive dye), turbidity
\end{itemize}

\subsubsection{Multi-Channel Coupling Implementation}

\textbf{Bioelectric channel}: Vesicles contain K$^+$ and Na$^+$ with ionophores. Membrane potential depends on internal ion concentrations. Neighbors in electrolytic contact sense each other's fields. Voltage-gated ionophore conformational changes provide threshold switching.

\textbf{Chemical channel}: Small molecules (H$^+$, metabolites) diffuse through shared aqueous medium between wells. Diffusion time sets the slow timescale.

\textbf{Mechanical channel}: Adjacent vesicles in physical contact transmit membrane tension. Osmotic pressure changes propagate through the array.

\textbf{Redox channel}: Redox-active species in the BZ chemistry (Ce$^{3+}$/Ce$^{4+}$, Fe$^{2+}$/Fe$^{3+}$) exchange electrons at membrane interfaces.

\subsubsection{Stimulus Protocol}

Stimulus configurations are defined by:
\begin{itemize}
\item UV intensity: Low / High (affects photochemistry)
\item Temperature: 20$^\circ$C / 40$^\circ$C (affects kinetics)
\item Ion composition: K$^+$-rich / Na$^+$-rich (affects membrane potential)
\item pH: 6.0 / 7.5 (affects reaction rates)
\item Forcing frequency: 0.1 Hz / 1 Hz (affects entrainment)
\end{itemize}

With 5 binary parameters, 32 distinct stimulus configurations are available.

\subsubsection{Pattern Readout}

Collective coordination patterns are read via:
\begin{itemize}
\item Spatial pattern of pH indicator fluorescence
\item Spatial pattern of voltage-sensitive dye signal
\item Temporal sequence of pattern states over multiple forcing cycles
\end{itemize}

A \textbf{symbol} is defined by the stable collective pattern (e.g., ``all vesicles in phase'' vs. ``alternating high/low'' vs. ``spiral wave''). A \textbf{character} is a sequence of symbols over multiple cycles.

\subsubsection{Computational Examples}

Numerical simulation of 61-vesicle hexagonal array with multi-channel coupling yields:

\begin{center}
\begin{tabular}{lcccc}
\toprule
\textbf{Stimulus Config} & \textbf{Pattern} & \textbf{Symbol} & \textbf{Stability} & \textbf{Repro} \\
\midrule
UV-Hi, T-Lo, K$^+$, pH-7.5, 1Hz & Uniform high & $S_0$ & 98\% & 100\% \\
UV-Hi, T-Hi, Na$^+$, pH-6.0, 0.1Hz & Alternating & $S_1$ & 97\% & 100\% \\
UV-Lo, T-Lo, K$^+$, pH-6.0, 1Hz & Spiral & $S_2$ & 96\% & 100\% \\
UV-Lo, T-Hi, Na$^+$, pH-7.5, 0.1Hz & Clustered & $S_3$ & 95\% & 100\% \\
\bottomrule
\end{tabular}
\end{center}

Summary:
\begin{itemize}
\item 32 stimulus configurations $\to$ 32 unique stable patterns (no collisions)
\item 32/32 configurations achieve 100\% reproducibility
\item Separation ratio (between-class / within-class variance): 335,000$\times$
\item Environment--attractor correlation: 0.72 (geometry-preserving)
\item Genuine temporal variation: 4-symbol sequences show different symbols across forcing cycles
\item Discretization emerges from substrate competition (lateral inhibition), not external logic
\item Diversity from spatially heterogeneous stimulus field (center-edge, top-bottom, left-right gradients)
\item Physics-based decoder (receiver colony) achieves 100\% accuracy without machine learning
\end{itemize}

\subsection{Second Embodiment: Droplet-in-Oil Array}

Water-in-oil droplets (50--100 $\mu$m) arranged in 2D array on hydrophobic surface. Coupling via:
\begin{itemize}
\item Bioelectric: Ion exchange through thin oil layer
\item Chemical: Lipophilic messengers partitioning between phases
\item Mechanical: Droplet deformation propagating through array
\end{itemize}

\subsection{Third Embodiment: Gel-Encapsulated Compartments}

Alginate or agarose beads (1--5 mm) containing oscillatory chemistry, embedded in gel matrix. Coupling via diffusion through gel. Slower dynamics, larger scale, easier to instrument.

\subsection{Verification Method}

The system passes correctness criteria if:
\begin{enumerate}
\item \textbf{Collective patterns}: Readout shows spatially organized coordination, not random noise
\item \textbf{Multi-stability}: $\geq$32 distinct stable patterns observed across stimulus conditions
\item \textbf{Reproducibility}: Same stimulus $\to$ same pattern $>$70\% of trials
\item \textbf{Game-theoretic stability}: Perturbation of single compartment does not destabilize collective pattern
\item \textbf{No pre-programming}: Pattern-symbol assignment derived from observation, not designed
\item \textbf{No biological material}: 16S rRNA qPCR negative
\end{enumerate}

%==============================================================================
\section{Claims}
%==============================================================================

\subsection*{Method Claims}

\textbf{Claim 1.} A method for generating digital codes from abiotic chemistry, comprising:
\begin{enumerate}
\item[(a)] providing an array of chemical compartments, each compartment containing reaction dynamics with at least 10 interacting chemical species;
\item[(b)] establishing coupling between adjacent compartments via at least one signaling channel selected from: bioelectric signaling, chemical signaling, mechanical signaling, or redox signaling;
\item[(c)] applying external stimulus to the array to drive the system away from equilibrium;
\item[(d)] allowing the array to settle into a stable collective coordination pattern; and
\item[(e)] reading the collective pattern as a digital symbol.
\end{enumerate}

\textbf{Claim 2.} The method of Claim 1, wherein coupling is established via at least two distinct signaling channels operating on different timescales.

\textbf{Claim 3.} The method of Claim 1, wherein bioelectric signaling comprises ion gradients across compartment membranes and voltage-dependent conformational changes.

\textbf{Claim 4.} The method of Claim 1, wherein chemical signaling comprises diffusion of small molecules through a shared medium.

\textbf{Claim 5.} The method of Claim 1, wherein mechanical signaling comprises membrane tension propagation between adjacent compartments.

\textbf{Claim 6.} The method of Claim 1, wherein redox signaling comprises electron transfer between compartments via redox-active species.

\textbf{Claim 7.} The method of Claim 1, wherein the compartments comprise lipid vesicles.

\textbf{Claim 8.} The method of Claim 1, wherein the compartments comprise water-in-oil droplets.

\textbf{Claim 9.} The method of Claim 1, wherein the compartments comprise gel-encapsulated reaction volumes.

\textbf{Claim 10.} The method of Claim 1, wherein the external stimulus comprises one or more of: UV irradiation, temperature change, ion composition change, pH change, or periodic forcing.

\textbf{Claim 10a.} The method of Claim 1, wherein the external stimulus comprises a spatially heterogeneous field such that different compartments within the array are exposed to differing stimulus intensities, thereby breaking symmetry and increasing the number of accessible coordination equilibria.

\textbf{Claim 11.} The method of Claim 1, wherein characters are formed by reading symbols at multiple successive time points.

\textbf{Claim 12.} The method of Claim 1, wherein the stable coordination pattern is a game-theoretic equilibrium in which each compartment's boundary state is a best response to its neighbors' boundary states.

\subsection*{System Claims}

\textbf{Claim 13.} A system for generating digital codes from abiotic chemistry, comprising:
\begin{enumerate}
\item[(a)] an array of chemical compartments, each containing high-dimensional reaction dynamics;
\item[(b)] coupling mechanisms establishing communication between adjacent compartments via at least one signaling channel;
\item[(c)] stimulus delivery apparatus for driving the array away from equilibrium;
\item[(d)] pattern readout apparatus configured to detect collective coordination states across the array; and
\item[(e)] symbol assignment logic associating stable patterns with digital symbols.
\end{enumerate}

\textbf{Claim 14.} The system of Claim 13, wherein the coupling mechanisms comprise ionophores enabling bioelectric communication between compartments.

\textbf{Claim 15.} The system of Claim 13, wherein the pattern readout apparatus comprises imaging of spatially-resolved indicator signals.

\textbf{Claim 16.} The system of Claim 13, wherein the array comprises a microfluidic chip with wells arranged in a hexagonal pattern.

\textbf{Claim 17.} The system of Claim 13, further comprising multiple readout channels for detecting patterns via different indicator modalities.

\subsection*{Coordination Claims}

\textbf{Claim 18.} The method of Claim 1, wherein the digital code emerges as a coordination equilibrium between compartments rather than from projection of a single compartment's dynamics.

\textbf{Claim 19.} The method of Claim 1, wherein the code is robust to perturbation of individual compartments due to game-theoretic stability of the collective pattern.

\textbf{Claim 20.} The method of Claim 1, wherein distinct stable coordination patterns correspond to distinct symbols, and the number of distinguishable symbols is at least 32.

\textbf{Claim 20a.} The method of Claim 1, wherein discretization of continuous dynamics into discrete symbols occurs via substrate competition (lateral inhibition), wherein output channels within each compartment compete for shared metabolic resources, forcing winner-take-all dynamics that create emergent digitality without external logic.

\subsection*{Application Claims}

\textbf{Claim 21.} Use of the method of Claim 1 for generating encoding tables mapping stimulus conditions to output symbols.

\textbf{Claim 22.} Use of the system of Claim 13 as an unconventional computing substrate.

\textbf{Claim 23.} Use of the system of Claim 13 as a physical unclonable function (PUF) for cryptographic applications, wherein the specific pattern-symbol mapping depends on fabrication variations.

\textbf{Claim 24.} Use of the method of Claim 1 for investigating the origin of biological codes.

\textbf{Claim 25.} Use of the method of Claim 1 for demonstrating that communication precedes information storage in the evolution of coded systems.

\end{document}
