\documentclass[11pt]{article}
\usepackage[margin=1in]{geometry}
\usepackage{hyperref}

\begin{document}

\noindent Ian Todd\\
Sydney Medical School\\
University of Sydney\\
Sydney, NSW 2006, Australia\\
itod2305@uni.sydney.edu.au

\vspace{1em}
\noindent\today

\vspace{1em}
\noindent Dear Editors of \textit{Discover Life},

\vspace{1em}
I am pleased to submit ``From Chemistry to Ecology: Codes and Predation Emerge from Coherence Constraints in Protocell Networks'' for consideration.

\subsection*{The Gap This Paper Addresses}

Decades of experimental work have demonstrated prebiotic nucleotide synthesis (Sutherland, Powner) and protocell assembly (Szostak, Chen). These are major achievements in \textit{chemistry}. But they do not address the \textit{coding} problem: how did discrete symbolic codes emerge from continuous chemical processes? The assignment problem---why UUU encodes phenylalanine---remains unexplained.

We propose a mechanism: \textbf{substrate competition}. When multiple output channels compete for a shared resource pool, winner-take-most dynamics discretize continuous inputs without any external digitizer. This is not engineered thresholding; it is the physics of competitive binding (Hill kinetics, $h > 1$).

\subsection*{The Theoretical Contribution}

The standard origin-of-life narrative assumes:
\begin{quote}
\textit{chemistry $\to$ molecules $\to$ replication $\to$ cells $\to$ ecology $\to$ cooperation}
\end{quote}

We invert this: \textit{social dynamics preceded molecular codes}:
\begin{quote}
\textit{compartments $\to$ coordination pressure $\to$ ecological differentiation $\to$ codes}
\end{quote}

The ecological dynamics we observe in mature biology---predator-prey relationships, competition, coordination costs---are not late additions. They are the \textit{drivers} of code emergence. This reframing is, to our knowledge, novel.

\subsection*{Key Results}

\textbf{Part I: Code Emergence}
\begin{itemize}
    \item 32 environments $\to$ 32 distinguishable codes (no collisions)
    \item 98\% decoding accuracy without learning---same chemistry decodes
    \item Ablations confirm substrate competition is essential ($h=1$ destroys codes)
    \item Information-theoretic analysis: 4.9 bits transmitted vs 5.0 bit capacity
\end{itemize}

\textbf{Part II: Ecological Differentiation}
\begin{itemize}
    \item Coordination ceiling derived from signal propagation physics
    \item Predator-prey dynamics emerge without specialized machinery
    \item Small coherent groups extract from large incoherent aggregates
\end{itemize}

\subsection*{What This Paper Does Not Claim}

We do not claim to have solved the origin of the genetic code. We provide a \textit{mechanism}---substrate competition in coupled compartments---that produces discrete codes from continuous chemistry. Whether this specific mechanism operated in early Earth conditions is an empirical question requiring experimental verification.

\subsection*{Relevance to Discover Life}

This manuscript addresses the journal's core interest: mechanisms underlying life's origins. The theoretical framework connects prebiotic chemistry (well-established) to ecological dynamics (well-observed) through a specific physical mechanism (testable). Experimental predictions are provided.

All simulation code is available at \url{https://github.com/todd866/protocell-codes}.

I confirm this manuscript is original and not under consideration elsewhere.

\vspace{1em}
\noindent Sincerely,

\vspace{2em}
\noindent Ian Todd\\
ORCID: 0009-0002-6994-0917

\end{document}
