\documentclass[11pt]{letter}
\usepackage[margin=1in]{geometry}
\usepackage[hidelinks]{hyperref}

\signature{Ian Todd\\Sydney Medical School\\University of Sydney}
\address{Sydney Medical School\\University of Sydney\\Sydney, NSW, Australia\\itod2305@uni.sydney.edu.au}

\begin{document}

\begin{letter}{Editorial Office\\Discover Life}

\opening{Dear Editors,}

I am pleased to submit ``Why Abiogenesis Experiments Produce Building Blocks But Not Codes: An Effective Dimensionality Threshold'' for consideration at \textit{Discover Life}.

\textbf{The puzzle.} Seventy years of abiogenesis research have produced amino acids, nucleotides, lipid vesicles, and ribozymes---but no codes. Why?

\textbf{The answer.} Code emergence requires sufficient \textit{effective} dimensionality ($D_{\text{eff}}$)---not merely many chemical species, but species whose dynamics span orthogonal information channels. I demonstrate a threshold at $D_{\text{eff}} > 1$: collapsed dynamics ($D_{\text{eff}} = 1$) achieve $>$80\% accuracy in only 35\% of cases, while diverse dynamics ($D_{\text{eff}} > 1$) achieve it in 71\%. The mechanism is substrate competition---competitive binding normalizes outputs into distinguishable patterns. Ablations show that graded competition works as well as cooperative binding; what matters is competitive allocation, not winner-take-most amplification.

\textbf{Why this matters.} This reframes a negative result (no codes in 70 years) as a positive prediction. Experiments were designed for mechanistic clarity, which means \textit{minimizing} orthogonal pathways---exactly the opposite of what code emergence requires. The key is not ``add more species'' but ``create functionally orthogonal pathways.'' The formose reaction, with autocatalytic structure creating divergent dynamics, may approach code-emergence conditions.

\textbf{Why this journal.} \textit{Discover Life} has published foundational work on prebiotic chemistry, autocatalytic sets, and protocell dynamics. This paper bridges these traditions by identifying effective dimensionality as the key variable connecting chemical complexity to functional organization.

The manuscript is approximately 4,500 words with two figures. All simulation code is publicly available. The work has not been submitted elsewhere.

\closing{Sincerely,}

\end{letter}
\end{document}
