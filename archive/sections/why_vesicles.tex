%==============================================================================
\section{Why Vesicles? Prebiotic Plausibility of Coupled Compartments}
%==============================================================================

Our model posits networks of coupled protocellular compartments as the substrate for code emergence. This section addresses whether such structures could have existed prebiotically, and why vesicles are the most plausible candidate.

\subsection{Evidence for Prebiotic Vesicle Formation}

Lipid vesicles form spontaneously under conditions consistent with early Earth:

\begin{enumerate}
    \item \textbf{Meteoritic delivery}: Amphiphilic molecules capable of forming membranes have been identified in carbonaceous chondrites, including the Murchison meteorite \cite{deamer2005chemistry,dworkin2001selfassembling}. These include fatty acids, alcohols, and polycyclic aromatic hydrocarbons with membrane-forming properties.

    \item \textbf{Hydrothermal synthesis}: Simulations of hydrothermal vent conditions produce amphiphilic molecules from simple precursors (CO$_2$, H$_2$) via Fischer-Tropsch-type reactions \cite{mccollom1999lipid,rushdi2001lipid}. Alkaline vents provide both the chemical gradients and the confined spaces conducive to protocell formation \cite{lane2010energetics,martin2008hydrothermal}.

    \item \textbf{Spontaneous assembly}: Above critical micelle concentrations, fatty acids spontaneously form bilayer vesicles in aqueous solution. Unlike phospholipids, short-chain fatty acids (C8--C12) form membranes that are permeable enough to allow exchange of small molecules while maintaining compartmentalization \cite{mansy2008model,budin2009expanding}.

    \item \textbf{Growth and division}: Szostak and colleagues have demonstrated that fatty acid vesicles can grow by incorporating additional amphiphiles from the environment and divide under mild shear forces---without enzymatic machinery \cite{zhu2009coupled,budin2011physical}. This provides a plausible pathway from chemistry to self-reproducing compartments.
\end{enumerate}

\subsection{Alternative Compartmentalization Substrates}

We acknowledge several alternative models for prebiotic compartmentalization:

\begin{center}
\begin{tabular}{lp{4.5cm}p{4.5cm}}
\toprule
\textbf{Substrate} & \textbf{Advantages} & \textbf{Limitations} \\
\midrule
Coacervates & Form spontaneously from polymers; concentrate molecules; Oparin's original model & No defined boundary; less structured; unclear heredity mechanism \\
\addlinespace
Hydrothermal vent pores & Mineral-catalyzed chemistry; natural gradients; fixed compartments & Geometry fixed by rock; cannot divide or evolve independently \\
\addlinespace
Mineral surfaces & Catalytic activity; template-capable (clays); concentrate reactants & Two-dimensional; no encapsulation; limited to surface chemistry \\
\addlinespace
Ice channels & Concentrate solutes via eutectic exclusion; stabilize polymers & Narrow temperature range; episodic availability \\
\addlinespace
Phase-separated droplets & Membraneless organelles exist in modern cells; form via liquid-liquid separation & No clear prebiotic formation pathway; less stable than vesicles \\
\bottomrule
\end{tabular}
\end{center}

Each substrate can provide compartmentalization, but they differ in whether they support the \textit{full set} of requirements for code emergence via coordination.

\subsection{Requirements for Coupled Compartment Dynamics}

Our framework requires compartments that satisfy five conditions:

\begin{enumerate}
    \item \textbf{Spatial separation}: Distinct interior volumes that can maintain different chemical states.

    \item \textbf{Semi-permeability}: Boundaries that allow selective exchange of signals (ions, small molecules, electrical potentials) while retaining macromolecules and maintaining gradients.

    \item \textbf{Internal chemistry}: Capacity to support non-equilibrium reaction dynamics---the ``high-dimensional reservoir'' that generates complex internal states.

    \item \textbf{Neighbor coupling}: Physical contact or proximity enabling boundary-to-boundary signaling between adjacent compartments.

    \item \textbf{Scalability}: Ability to form extended networks (arrays, chains, clusters) where collective dynamics can emerge.
\end{enumerate}

\noindent\textbf{Vesicles uniquely satisfy all five requirements}:

\begin{itemize}
    \item \textit{Spatial separation}: Bilayer membranes create distinct aqueous compartments.
    \item \textit{Semi-permeability}: Fatty acid membranes are permeable to small molecules and ions; membrane proteins (or prebiotic analogs) can create selective channels \cite{mansy2008model}.
    \item \textit{Internal chemistry}: Encapsulated reaction networks can operate far from equilibrium, driven by transmembrane gradients.
    \item \textit{Neighbor coupling}: Vesicles in contact can exchange signals via membrane fusion, hemifusion, gap junction-like connections, or diffusion through shared aqueous phases. Bioelectric signaling (voltage gradients across membranes) provides a fast coordination channel \cite{levin2021bioelectric}.
    \item \textit{Scalability}: Vesicles spontaneously form aggregates, chains, and two-dimensional arrays when concentrated on surfaces or in confined geometries \cite{karlsson2001networks,bolognesi2018sculpting}.
\end{itemize}

Coacervates satisfy (1), (3), and partially (2), but lack the structured boundaries needed for (4) and (5). Mineral pores satisfy (1)--(4) but fail (5)---they cannot form mobile, evolvable networks. Vesicles are the minimal prebiotic structure that supports the full coordination-to-code pathway.

\subsection{The Mechanism Is Substrate-Agnostic}

While we focus on vesicles for concreteness and experimental tractability, we emphasize that the \textit{mechanism}---codes emerging from coordination equilibria between coupled high-dimensional systems---is substrate-agnostic.

Any system satisfying the five requirements above could, in principle, generate symbolic communication via the pathway we describe:
\begin{quote}
Coupled dynamics $\to$ coordination equilibria $\to$ stable boundary symbols $\to$ compressed codes
\end{quote}

This generality is important: it means the transition from chemistry to codes is not contingent on a specific molecular implementation but emerges from \textit{geometric and dynamical constraints} on coupled compartments. Different early Earth environments may have hosted different compartmentalization substrates, all potentially capable of bootstrapping proto-symbolic communication.

\subsection{Experimental Accessibility}

A practical advantage of the vesicle model is experimental tractability. Modern microfluidic techniques allow:
\begin{itemize}
    \item Fabrication of vesicle arrays with controlled topology \cite{karlsson2001networks}
    \item Loading with oscillatory chemistry (e.g., BZ reaction, glycolytic oscillators)
    \item Measurement of inter-vesicle coupling via fluorescence, pH indicators, or voltage-sensitive dyes
    \item Systematic variation of coupling strength, membrane composition, and network geometry
\end{itemize}

The predictions of our model---that coupled vesicle networks will spontaneously develop reproducible, discrete, decodable boundary patterns---are directly testable with existing laboratory techniques.

%--- References to add to bibliography ---
% \bibitem{deamer2005chemistry} D.W. Deamer and J.P. Dworkin. Chemistry and physics of primitive membranes. Top. Curr. Chem., 259:1--27, 2005.
% \bibitem{dworkin2001selfassembling} J.P. Dworkin et al. Self-assembling amphiphilic molecules: Synthesis in simulated interstellar/precometary ices. PNAS, 98:815--819, 2001.
% \bibitem{mccollom1999lipid} T.M. McCollom et al. Lipid synthesis under hydrothermal conditions. Orig. Life Evol. Biosph., 29:153--166, 1999.
% \bibitem{rushdi2001lipid} A.I. Rushdi and B.R.T. Simoneit. Lipid formation by aqueous Fischer-Tropsch-type synthesis. Orig. Life Evol. Biosph., 31:103--118, 2001.
% \bibitem{martin2008hydrothermal} W. Martin and M.J. Russell. On the origin of biochemistry at an alkaline hydrothermal vent. Phil. Trans. R. Soc. B, 362:1887--1926, 2007.
% \bibitem{mansy2008model} S.S. Mansy et al. Template-directed synthesis of a genetic polymer in a model protocell. Nature, 454:122--125, 2008.
% \bibitem{budin2009expanding} I. Budin and J.W. Szostak. Expanding roles for diverse physical phenomena during the origin of life. Annu. Rev. Biophys., 39:245--263, 2010.
% \bibitem{zhu2009coupled} T.F. Zhu and J.W. Szostak. Coupled growth and division of model protocell membranes. JACS, 131:5705--5713, 2009.
% \bibitem{budin2011physical} I. Budin and J.W. Szostak. Physical effects underlying the transition from primitive to modern cell membranes. PNAS, 108:5249--5254, 2011.
% \bibitem{karlsson2001networks} M. Karlsson et al. Networks of nanotubes and containers. Nature, 409:150--152, 2001.
% \bibitem{bolognesi2018sculpting} G. Bolognesi et al. Sculpting and fusing biomimetic vesicle networks. Nat. Commun., 9:1882, 2018.
