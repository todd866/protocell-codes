\documentclass[12pt]{article}
\usepackage[margin=1in]{geometry}
\usepackage{amsmath,amssymb}
\usepackage{graphicx}
\usepackage{tikz}
\usetikzlibrary{arrows.meta,shapes,positioning}
\usepackage{booktabs}
\usepackage{hyperref}
\usepackage{xcolor}
\usepackage{enumitem}

\title{\textbf{Ecology Without Teeth: Predator-Prey Dynamics from Coherence Constraints in Protocell Networks}}

\author{Ian Todd\\
Sydney Medical School\\
University of Sydney\\
Sydney, NSW, Australia\\
\texttt{itod2305@uni.sydney.edu.au}}

\date{}

\begin{document}

\maketitle

%==============================================================================
\begin{abstract}
%==============================================================================
We show that predator-prey dynamics emerge generically from a single constraint: coherence is metabolically expensive and scales superlinearly with group size. Using a minimal model of coupled protocell networks, we demonstrate a three-phase evolutionary arc: (1) a \textit{size regime} where incoherent aggregates compete on headcount, (2) a \textit{coherence regime} where small coordinated groups outcompete large ones via coordination multipliers, and (3) a \textit{parasitism regime} where coherence caps on scale force high-coherence groups to grow through extraction rather than expansion. The resulting ecology---small coherent ``predators'' raiding large incoherent ``prey''---requires no specialized predation machinery. Predation is asymmetric extraction enabled by coherence differentials. This framework addresses a gap in abiogenesis theory: why early life did not homogenize into synchronous goo. The metabolic ceiling on coherence prevents homogenization, creates ecological niches, and drives a complexity ratchet. We argue that ecology is not an add-on to life but rather the stabilizer that makes abiogenesis viable.
\end{abstract}

\medskip
\noindent\textbf{Keywords:} origin of life; predator-prey dynamics; coherence; protocells; ecological phase transitions; complexity ratchet

%==============================================================================
\section{Introduction: The Homogenization Problem}
%==============================================================================

Abiogenesis theories face a persistent objection: if coherent, coordinated dynamics can arise spontaneously (as proposed in \cite{todd2026codes}), why doesn't everything simply synchronize into one homogeneous blob? What prevents the entropy-minimizing outcome of global coherence?

We propose that the answer is \textbf{metabolic}: coherence is expensive, and the cost scales superlinearly with group size. This creates an inherent ceiling on coherent aggregates, which in turn creates ecological differentiation---and predator-prey dynamics---without any specialized predation machinery.

The logic is straightforward:
\begin{enumerate}
    \item Coherent coordination requires energy (coupling maintenance, error correction, synchronization costs)
    \item These costs grow faster than linearly with group size ($\text{Cost} \sim k N^\gamma C^\zeta$, with $\gamma > 1$)
    \item Therefore, high-coherence groups are forced toward small $N$
    \item Small coherent groups can concentrate force effectively against large incoherent ones
    \item The growth strategy for coherent groups becomes extraction, not expansion
\end{enumerate}

This is predation without teeth---asymmetric resource transfer driven by coherence differentials.

\subsection{Connection to Abiogenesis}

Our companion paper \cite{todd2026codes} demonstrated that symbolic codes emerge spontaneously as coordination interfaces between coupled protocell networks. That paper established \textit{that} coherence can arise; this paper addresses \textit{why it persists} without homogenizing.

The two papers form a logical lock:
\begin{itemize}
    \item \textbf{Paper A} (Codes as Coordination): Coherence-seeking colonies can form at all without genes
    \item \textbf{Paper B} (this paper): Those colonies don't dissolve or homogenize---metabolic ceilings create ecology
\end{itemize}

Paper B retroactively validates Paper A by showing that the attractors it predicts are ecologically stable, not just mathematically possible.

\subsection{Outline}

Section~\ref{sec:model} defines the model explicitly. Section~\ref{sec:tradeoff} develops the core trade-off between coherence and scale. Section~\ref{sec:phases} describes the three-phase evolutionary arc. Section~\ref{sec:ecology} shows how predator-prey dynamics emerge, including the explicit extraction operator. Section~\ref{sec:simulations} presents simulation results. Section~\ref{sec:discussion} discusses implications and extensions.

%==============================================================================
\section{Model Definition}
\label{sec:model}
%==============================================================================

We consider a population of protocell networks (``colonies'') competing for resources in a shared environment.

\subsection{State Variables}

Each colony $i$ is characterized by:
\begin{itemize}
    \item $N_i \in [1, N_{\max}]$: number of compartments (size)
    \item $C_i \in [0, 1]$: coherence (Kuramoto order parameter of internal oscillators)
    \item $R_i$: accumulated resources
\end{itemize}

\subsection{Environment}

Resources arrive at rate $\rho$ per unit area. Colonies occupy space proportional to $N_i$ and collect resources proportionally:
\begin{equation}
\dot{R}_i^{\text{gather}} = \rho \cdot N_i
\end{equation}

\subsection{Coordination Multiplier}

Coherence amplifies effective resource use via a coordination multiplier:
\begin{equation}
\phi(C) = 1 + a(C - C^*)^m \cdot \mathbf{1}_{C > C^*}
\label{eq:phi}
\end{equation}
where $C^* \approx 0.3$ is the coordination threshold and $m > 1$ makes the benefit superlinear above threshold. We use $a = 2$, $m = 2$ throughout.

\subsection{Coherence Cost}

Maintaining coherence costs energy:
\begin{equation}
\text{Cost}_i = k \cdot N_i^\gamma \cdot C_i^\zeta
\label{eq:cost}
\end{equation}
with $\gamma > 1$ (superlinear in size) and $\zeta \geq 1$. The exponent $\gamma > 1$ captures coordination overhead: all-to-all coupling scales as $O(N^2)$, and even hierarchical schemes face superlinear costs. We use $\gamma = 1.5$, $\zeta = 1$, $k = 0.1$ as baseline parameters.

\subsection{Net Fitness}

Instantaneous fitness:
\begin{equation}
W_i = N_i \cdot \bar{R} \cdot \phi(C_i) - k \cdot N_i^\gamma \cdot C_i^\zeta
\label{eq:fitness}
\end{equation}
where $\bar{R}$ is mean resource density.

\subsection{Selection}

Each generation:
\begin{itemize}
    \item Colonies with $W_i < 0$ die
    \item Surviving colonies reproduce with probability $\propto \max(W_i, 0)$
    \item Offspring inherit $(N, C)$ with bounded mutation: $\Delta N \sim \mathcal{N}(0, \sigma_N^2)$, $\Delta C \sim \mathcal{N}(0, \sigma_C^2)$, reflected at boundaries
\end{itemize}

%==============================================================================
\section{The Coherence-Scale Trade-off}
\label{sec:tradeoff}
%==============================================================================

\subsection{Coherence as Coordination Multiplier}

In the absence of coherence ($C = 0$), fitness reduces to:
\begin{equation}
W_0 = N \cdot \bar{R} - k \cdot N^\gamma \cdot 0 = N \cdot \bar{R}
\end{equation}

With coherence above threshold, the multiplier $\phi(C) > 1$ amplifies effectiveness. A coordinated group of 10 can outperform an uncoordinated group of 100 if $\phi(C)$ is sufficiently large.

This is well-established in the collective behavior literature \cite{couzin2009collective,sumpter2010collective}: coordinated groups achieve outcomes unavailable to equivalent numbers of uncoordinated individuals.

\subsection{The Metabolic Ceiling}

But coherence costs grow superlinearly with size (Eq.~\ref{eq:cost}). Maximizing fitness with respect to $N$ at fixed $C$:
\begin{equation}
\frac{\partial W}{\partial N} = \bar{R} \cdot \phi(C) - k \gamma N^{\gamma-1} C^\zeta = 0
\end{equation}
yields optimal size:
\begin{equation}
N^* = \left( \frac{\bar{R} \cdot \phi(C)}{k \gamma C^\zeta} \right)^{1/(\gamma-1)}
\end{equation}

Since $\gamma > 1$, higher $C$ implies lower $N^*$. \textbf{High coherence forces small size.}

\subsection{Two Stable Strategies}

This creates two stable strategies:
\begin{itemize}
    \item \textbf{Type B (big/weak)}: Large $N$, low $C$. Wins by occupying space and capturing resources through coverage.
    \item \textbf{Type S (small/coherent)}: Small $N$, high $C$. Wins by concentrated, coordinated action.
\end{itemize}

Neither can invade the other's niche through pure competition. This is the foundation of ecological differentiation.

%==============================================================================
\section{Three-Phase Evolutionary Arc}
\label{sec:phases}
%==============================================================================

Starting from random initial conditions, selection drives the system through three distinct phases:

\subsection{Phase 1: Size Regime}

When coherence mechanisms are absent or weak ($C \approx 0$ for all), fitness reduces to $W \approx N \cdot \bar{R}$.

Selection favors aggregation. Big incoherent blobs outcompete small incoherent blobs. This is the ``primordial soup'' regime---undifferentiated masses competing for resources through sheer coverage.

\subsection{Phase 2: Coherence Regime}

Once coupling mechanisms exist (as developed in \cite{todd2026codes}), mutations can increase $C$. Because $\phi(C)$ is superlinear above threshold, even modest coherence provides large fitness gains.

A small coherent group can outcompete a large incoherent one if $\phi(C_{\text{small}}) > N_{\text{large}}/N_{\text{small}}$.

Selection now favors coordination. Groups that achieve phase-locking dominate those that don't.

\subsection{Phase 3: Parasitism Regime}

But coherence has a ceiling (Eq.~\ref{eq:cost}). Groups that maximize $C$ are forced toward small $N$.

These small coherent groups face a problem: they can't grow by expansion (metabolic ceiling). But they \textit{can} extract resources from large incoherent groups.

This is the transition to predation.

%==============================================================================
\section{Ecology from Coherence Differentials}
\label{sec:ecology}
%==============================================================================

\subsection{The Extraction Operator}

We now make predation explicit. Each generation, predator-prey encounters occur (either locally in space or via random mixing). A raid is resolved as follows:

\textbf{Raid success probability:}
\begin{equation}
p_{\text{raid}} = \sigma\left( \kappa_C \log\frac{C_S}{C_B} - \kappa_N \log\frac{N_B}{N_S} \right)
\label{eq:raid}
\end{equation}
where $\sigma(\cdot)$ is the sigmoid function and $\kappa_C, \kappa_N > 0$ are sensitivity parameters.

\textbf{Interpretation:} Coherence advantage ($C_S > C_B$) increases success; size disadvantage ($N_S < N_B$) decreases it. The log-ratio form ensures scale invariance.

\textbf{Resource transfer:} If the raid succeeds, the predator (S) extracts fraction $f$ of the prey's (B) resources:
\begin{equation}
\Delta R_S = +f \cdot R_B, \quad \Delta R_B = -f \cdot R_B
\end{equation}

\textbf{Metabolic ceiling enforcement:} Predators cannot increase $N$ beyond their coherence-determined ceiling. Surplus resources convert to reproduction rate, not size growth.

This operator is minimal but sufficient: it produces predator-prey oscillations without any ``teeth.'' Predation is asymmetric extraction enabled by coherence differentials.

\subsection{The Two-Attractor Landscape}

The system settles into a two-attractor ecology:

\begin{center}
\begin{tabular}{lcc}
\toprule
Property & Type B (Prey) & Type S (Predator) \\
\midrule
Size $N$ & Large & Small \\
Coherence $C$ & Low & High \\
Metabolic rate & Low & High \\
Carrying capacity & High & Low \\
Growth strategy & Space occupation & Targeted extraction \\
\bottomrule
\end{tabular}
\end{center}

This is predator-prey dynamics without any predation-specific machinery. The structure emerges from the coherence-scale trade-off alone.

\subsection{Population Dynamics}

The extraction operator (Eq.~\ref{eq:raid}) generates coupled population dynamics qualitatively similar to Lotka-Volterra:
\begin{itemize}
    \item Abundant prey $\to$ predator population grows (more extraction opportunities)
    \item High predation $\to$ prey population declines
    \item Scarce prey $\to$ predator population crashes (extraction fails)
    \item Low predation $\to$ prey recovers
\end{itemize}

The mechanism differs from classical predator-prey models: predators don't consume prey biomass directly. They extract resources through coordinated raids enabled by coherence advantage.

\subsection{Why Homogenization Fails}

This framework explains why early life didn't collapse into synchronous goo:

\begin{enumerate}
    \item If everyone becomes highly coherent, metabolic costs explode (ceiling binds)
    \item If everyone becomes large and incoherent, coherent raiders emerge (parasitism niche opens)
    \item The only stable state is \textbf{ecological differentiation}
\end{enumerate}

Homogenization is not evolutionarily stable. The trade-off guarantees diversity.

%==============================================================================
\section{Simulations}
\label{sec:simulations}
%==============================================================================

\subsection{Model Implementation}

We implement the model described in Section~\ref{sec:model} with the extraction operator from Eq.~\ref{eq:raid}. Each simulation runs for $T = 1000$ generations with $M = 500$ initial colonies.

\textbf{Parameters:} $\gamma = 1.5$, $\zeta = 1$, $k = 0.1$, $a = 2$, $m = 2$, $C^* = 0.3$, $\kappa_C = 2$, $\kappa_N = 1$, $f = 0.2$.

\subsection{Results}

[Figures to be generated from simulation code]

\textbf{Figure 1} (Phase diagram): Population distribution in $(N, C)$ space at $t = 0$ (random), $t = 100$ (clustering begins), $t = 1000$ (two-attractor structure). Viability boundary and metabolic ceiling shown.

\textbf{Figure 2} (Population dynamics): Predator and prey population counts over time, showing phase-shifted oscillations qualitatively similar to Lotka-Volterra dynamics.

\textbf{Figure 3} (Robustness): Two-attractor structure persists across parameter sweeps in $(\gamma, \zeta)$ space, with $\gamma > 1$ as the critical condition.

\subsection{Robustness}

We verified the regime structure across parameter sweeps:
\begin{itemize}
    \item $\gamma \in [1.1, 2.5]$: Two-attractor structure present for all $\gamma > 1$
    \item $\zeta \in [0.5, 2.0]$: Structure robust across range
    \item $k \in [0.01, 1.0]$: Structure preserved; only basin boundaries shift
\end{itemize}

The key requirement is $\gamma > 1$ (superlinear coordination cost). Below this threshold, large coherent groups dominate and ecology collapses to a single attractor.

%==============================================================================
\section{Discussion}
\label{sec:discussion}
%==============================================================================

\subsection{Implications for Abiogenesis}

Our framework suggests that \textbf{ecology is not an add-on to life}. It is the stabilizer that makes abiogenesis possible. Without metabolic ceilings on coherence, early life would have homogenized. Predator-prey dynamics are not a late elaboration---they are present from the first moment that coordinated groups exist, given superlinear coordination costs.

\subsection{Multicellularity}

The same trade-off operates at the transition to multicellularity. A multicellular organism is a high-coherence group that has internalized coordination costs. The metabolic ceiling explains why multicellular organisms can't grow indefinitely---and why large organisms are vulnerable to small, coordinated pathogens.

\subsection{Attention Economies}

The framework extends to modern contexts. Screen-based attention capture can be understood as semiotic predation: an external system extracting coherence from a target (e.g., a developing brain's self-stabilizing dynamics) through entrainment. The ``predator'' is the algorithm optimizing for engagement.

\subsection{Outlook: Semiotic Predation}

Once symbolic codes emerge (as in \cite{todd2026codes}), a new layer of predation becomes available: code spoofing, mimicry, and authentication arms races. This creates a \textbf{complexity ratchet}---code complexity increases monotonically as prey evolve defenses and predators evolve exploits. We leave simulation of semiotic predation to future work.

\subsection{Conclusion}

Predation does not require teeth. It requires coherence differentials. The metabolic ceiling on coordination creates ecological niches, drives complexity ratchets, and prevents the homogenization that would otherwise doom early life.

Ecology is not downstream of abiogenesis. Ecology is what makes abiogenesis stable.

%==============================================================================
\section*{Acknowledgments}
%==============================================================================

[To be added]

%==============================================================================
\bibliographystyle{plain}
\begin{thebibliography}{99}

\bibitem{todd2026codes}
Todd, I. (2026).
Codes as Coordination: How Coupled Protocells Generate Symbolic Communication Before Genes.
\textit{BioSystems} (in preparation).

\bibitem{todd2026intelligence}
Todd, I. (2026).
Intelligence as High-Dimensional Coherence: Observable Dimensionality Bounds on Autonomous Systems.
\textit{BioSystems} (under review).

\bibitem{couzin2009collective}
Couzin, I.D. (2009).
Collective cognition in animal groups.
\textit{Trends in Cognitive Sciences}, 13(1), 36--43.

\bibitem{sumpter2010collective}
Sumpter, D.J. (2010).
\textit{Collective Animal Behavior}.
Princeton University Press.

\end{thebibliography}

\end{document}
